\documentclass[conference]{IEEEtran}
\usepackage{cite}
\usepackage{amsmath,amssymb,amsfonts}
\usepackage{algorithm}
\usepackage{algorithmic}
\usepackage{graphicx}
\usepackage{textcomp}
\usepackage{xcolor}
\usepackage{multicol}
\usepackage{float}
\usepackage[spanish]{babel}

\def\BibTeX{{\rm B\kern-.05em{\sc i\kern-.025em b}\kern-.08em
    T\kern-.1667em\lower.7ex\hbox{E}\kern-.125emX}}
\begin{document}

\title{Esteganografía en Imágenes}
\author{\IEEEauthorblockN{Joaquín Pérez Araya}
\IEEEauthorblockA{\textit{Departamento de Ciencias de la Computación} \\
\textit{Universidad de Chile}\\
Santiago, Chile \\
joaquin.perez.a@ug.uchile.cl}}


\maketitle

\begin{abstract}
    
\end{abstract}
 

\section*{Introducción} % ***Así la cosa no me molesta con los numeritos***
    
    
\section*{Diseño e Implementación}
    \subsection*{Histogramas}
    \subsection*{Consultas}
	
	
\section*{Experimentación}
    

\section*{Conclusión}
    


\begin{thebibliography}{99}
\bibitem{DIS} Cheddad, A., Condell, J., Curran, K., \& Mc Kevitt, P. (2010). Digital image 
steganography: Survey and analysis of current methods. Signal Processing, 90(3), 727–752. 


\end{thebibliography}

\section*{Anexo}

\end{document}
